\documentclass[11pt]{article}
\usepackage{graphicx} % This lets you include figures
\usepackage{hyperref} % This lets you make links to web locations
\usepackage[margin=0.5in]{geometry}
\usepackage[rightcaption]{sidecap}
\usepackage{subcaption}
\usepackage{wrapfig}
\usepackage{float}
\usepackage{imakeidx}
\usepackage{indentfirst}
\makeindex
%---------------------------Do Not Edit Anything Above This Line!!------------------------

% edit the line below, if needed, to change the directory name for your image files.
\graphicspath{ {./images/} }



\begin{document}

%---------------------------Edit Content in the Box to Create the Title Page--------------
\begin{titlepage}
   \begin{center}
       \vspace*{1cm}
	   \Huge
       \textbf{Automated Testing Framework for a Distributed Disaster Response System}

       \vspace{0.5cm}
       \Large
       Project 2 \\
       9-30-2025 \\
   \end{center}

       \vspace{1.5cm}

\begin{table}[h!]
\centering
\begin{tabular}{|l|l|}
\hline
\textbf{Name} & \textbf{Email Address} \\ \hline
Nate Barner& nathaniel.barner603@topper.wku.edu\\ \hline
Kaden Hunt& kaden.hunt144@topper.wku.edu\\ \hline
\end{tabular}
\end{table}

%Latex Table Generator    
%https://www.tablesgenerator.com/     
        
\vspace{4in}

\centering        
CS 396 \\
Fall 2025\\
Project Organization Documentation

\end{titlepage}
%---------------------------Edit Content in the Box to Create the Title Page--------------


% No text here.


%---------------------------Do Not Edit Anything In This Box!!------------------------
%Table of contents and list of figures will be autogenerated by this section.
\newpage
\setcounter{page}{1}%
\cleardoublepage
\pagenumbering{gobble}
\tableofcontents
\cleardoublepage
\pagenumbering{arabic}
\clearpage
\newpage
\setcounter{page}{1}%
\cleardoublepage
\pagenumbering{gobble}
\listoffigures
\cleardoublepage
\pagenumbering{arabic}
\newpage
%---------------------------Do Not Edit Anything In This Box!!------------------------

% No text here.


%---------------------------Project Team's Organizational Approach------------------------------
\section{Project Team's Organizational Approach} %\section{} is used to create major section headers
%How/where did the group meet?  How often did you meet as an entire team? 
The project team consisted of two members: Kaden Hunt and Nate Barner. Tasks were divided based on 
each member’s skills and familiarity with the work. Kaden focused on setting up the CI workflow, managing the \texttt{requirements.txt} file, and creating a script to generate 
reports from artifacts. Nate worked on building the mock disaster response application and writing 
unit, integration, and system-level test cases. Both members contributed to documentation and final 
deliverables.

Because both members share a class schedule, the team was able to meet in person Monday through 
Friday for 20–30 minutes or longer, depending on workload. Additional coordination happened through 
text messages, which allowed quick updates, task assignments, and troubleshooting. Decisions were 
made collaboratively, and responsibilities were shared to ensure balanced contributions and steady 
progress toward project goals.


%---------------------------End Project Team's Organizational Approach------------------------------


% No text here.


%---------------------------Schedule Organization---------------------------------------------------
\section{Schedule Organization}
%Gantt charts cover the tasks/time commitments and estimations for the entire project.  We will have four iterations of the Gantt Chart, with iteration focusing on a specific sprint.
The schedule was organized to align work with the three-week timeline between assignment and 
submission. Early tasks included setting up the repository, CI pipeline, and dependency files. 
The middle phase was dedicated to test development for unit, integration, and system levels. The 
final phase focused on polishing documentation, preparing slides, and recording the demo video.
\subsection{Gantt Chart:}
%100 words to describe the focus for this sprint.
%Identify the location for the Gantt Chart.  Should be clearly labeled in the project directory.
The Gantt chart illustrates the sequence of tasks from initial setup through final submission and 
is located in the \texttt{docs/} folder of the repository. The chart highlights milestones such as 
repository creation, CI pipeline configuration, test development, documentation drafting, and demo 
preparation. The focus of this sprint was ensuring that all requirements were met and that every 
deliverable—including reports, presentation, and video—was ready before the September 30, 2025 
deadline.

%---------------------------End Schedule Organization---------------------------------------------------


% No text here.


%---------------------------Progress Visibility---------------------------------------------------
\section{Progress Visibility}
%Explain how each member of the group is progressing with assigned tasks and how that progress is shared with the group.  Examples:  how does the group assign tasks?  How to group members know tasks assigned to them?  How do group members communicate when assigned tasks are complete, need assistance, or waiting on other tasks to be completed first?

Progress throughout the project was maintained by dividing tasks according to each member’s 
strengths. Kaden Hunt focused on writing the \texttt{ci.yml} workflow file, To monitor progress, the team relied on daily class meetings and frequent text communication. 
These daily in-person check-ins provided a consistent opportunity to review what had been completed, 
what needed attention, and whether any issues required support. Text updates filled the gaps 
between meetings and kept tasks moving forward.  

Assignments were agreed on collaboratively, with responsibilities divided according to strengths. 
When a task was completed, it was communicated immediately so that dependent work could continue. 
This steady flow of updates and visibility ensured both members stayed aligned and reduced the risk 
of missed deadlines.
%---------------------------End Progress Visibility---------------------------------------------------


% No text here.


%---------------------------Risk Management--------------------------------------------------------------
\section{Risk Management}
%Use this section to describe the team's approach to risk management.
The most significant risks for this project came from the short timeline and the team’s limited 
experience with CI/CD tools. With only twenty-one days available, delays in configuring GitHub 
Actions or developing automated tests could have impacted the schedule.  

These risks were managed by investing time early in research. The team reviewed documentation, 
consulted templates, and used tutorials to build familiarity with pytest, coverage tools, and GitHub 
Actions. Dividing work clearly also reduced bottlenecks, since each member focused on specific 
responsibilities. Frequent check-ins surfaced issues quickly, and incremental testing allowed 
problems to be solved before they affected later stages. This approach helped the project stay on 
track and ensured that final deliverables were completed on time.

%---------------------------Risk Management--------------------------------------------------------------





%example image:  uncomment to show usage
%\begin{figure}[h]
%    \centering
%    \includegraphics[width=1\textwidth]{images/Add_non-music.png}
%    \caption{This is how you add non-music items.}
%    \label{fig16}
%\end{figure}


%example links:  uncomment to show usage.
%\url{https://www.youtube.com}
%\href{https://www.wku.edu/}{WKU Homepage}
%\footnote{You can put the link in a footnote like this.}

% Anything to the right of a percent sign will be ignored by LaTeX.
% You can use this to put notes to yourself.  



\end{document}
